\section{Proofs}

{\bf \cref{lem:equivalence_ment_reinforce}.} Using importance sampling in \cref{alg:erpg_ment_reinforce},
\begin{equation*}
    \nabla{(\rvtheta_t)} = \expectation\limits_{\rho \sim \rvpi_t}{ \left( \rvr_t(\rho) - \tau \log{\rvpi_t(\rho)} \right) \cdot \nabla{ \log{\rvpi_t(\rho)} } }, \quad \forall t \ge 1.
\end{equation*}
\begin{proof}
According to the gradient definition in \cref{alg:erpg_ment_reinforce}, $\forall t \ge 1$,
\begin{equation*}
\begin{split}
    \nabla{(\rvtheta_t)} &= \frac{d \left\{ \rvpi(\rvo_t)^\top \rvr_t - \tau  \rvpi(\rvo_t)^\top \log{\rvpi(\rvo_t)} \right\} }{d \rvtheta_t} \\
    &= \left( \frac{d \rvpi_t}{d \rvo_t} \right)^\top \left( \rvr_t - \tau \rvone - \tau \log{\rvpi_t} \right) \\
    &= \left( \frac{d \rvpi_t}{d \rvtheta_t} \right)^\top \left( \rvr_t - \tau \log{\rvpi_t} \right) \\
    &= \left( \frac{d \rvpi_t}{d \rvtheta_t} \right)^\top \left[ \sum\limits_{i \in [K]}{ \rvpi_t(i) \cdot \rvone_{ \left\{ \frac{\rvr_t(i) - \tau \log{\rvpi_t(i)} }{\rvpi_t(i)} \right\} } } \right] \quad \left( \text{importance sampling} \right) \\
    &= \sum\limits_{i \in [K]}{ \rvpi_t(i) \cdot \left[ \left( \frac{d \rvpi_t}{d \rvtheta_t} \right)^\top \rvone_{ \left\{ \frac{\rvr_t(i) - \tau \log{\rvpi_t(i)} }{\rvpi_t(i)} \right\} } \right] } \\
    &= \sum\limits_{i \in [K]}{ \rvpi_t(i) \cdot \left[ \frac{\rvr_t(i) - \tau  \log{\rvpi_t(i)} }{\rvpi_t(i)} \cdot \frac{d \rvpi_t(i)}{d \rvtheta_t}  \right] } \\
    &= \expectation\limits_{\rho \sim \rvpi_t}{ \left( \rvr_t(\rho) - \tau  \log{\rvpi_t(\rho)} \right) \cdot \nabla{ \log{\rvpi_t(\rho)} } }. \qedhere
\end{split}
\end{equation*}
\end{proof}


{\bf \cref{thm:logit_mean_approximation}.} Let $\rvr_1, \rvr_2, \dots, \rvr_t \in \left[ 0, 1  \right]^K$. Denote $\rvr_{1:t} \triangleq \sum\limits_{i=1}^{t}{\rvr_i}$, and $\hat{\rvr}_t \triangleq \frac{\rvr_{1:t}}{t}$. Consider the following policy gradient update in logit space,
\begin{equation*}
\begin{split}
    \rvo_{t+1} &= \rvo_t + \eta \cdot \nabla(\rvo_t),
\end{split}
\end{equation*}
where $\nabla(\rvo_t) \triangleq \frac{d \left\{ \rvpi(\rvo_t)^\top \rvr_t - \tau \rvpi(\rvo_t)^\top \log{\rvpi(\rvo_t)} \right\} }{d \rvo_t} $ is the policy gradient with respect to $\rvo_t$. If the learning rate $\eta$ at iteration $t$ satisfies
\begin{equation*}
\begin{split}
    \eta \ge \frac{ \frac{1}{t+1} + \frac{1}{(t+1)(t+C)} }{ \frac{1}{K \cdot \exp\left\{ \frac{1}{\tau} \right\} } \cdot \left( 1 - \frac{2 C \sqrt{K}}{\tau t} \right)  } = K \cdot \exp\left\{ \frac{1}{\tau} \right\} \cdot \left( 1 + \frac{1}{t+C} \right) \cdot \frac{t}{\left( t+1\right) \left( t - \frac{2C\sqrt{K}}{\tau} \right)} = O\left( \frac{1}{t} \right),
\end{split}
\end{equation*}
for some constant $C > 0$, then there exists a scalar $c_{t-1} \in \sR$, such that
\begin{equation*}
    \left\| \tau \rvo_{t} - \hat{\rvr}_{t-1} + c_{t-1} \rvone \right\| \le \frac{C \sqrt{K}}{t}, \quad \forall t \ge 1.
\end{equation*}
\begin{proof}
For any policy $\rvpi$, define $\rmH\left( \rvpi \right) \triangleq \Delta \left( \rvpi \right) - \rvpi \rvpi^\top$, where $\Delta \left( \rvpi \right) \in \sR^{K \times K}$ is a diagonal matrix, with $\rvpi$ as its diagonal. First, calculate the policy gradient with respect to logit at step $t$,
\begin{equation*}
\begin{split}
    \nabla(\rvo_t) &\triangleq \frac{d \left\{ \rvpi(\rvo_t)^\top \rvr_t - \tau \rvpi(\rvo_t)^\top \log{\rvpi(\rvo_t)} \right\} }{d \rvo_t} \\
    &= \rmH_t \left( \rvr_t - \tau \rvo_t \right),
\end{split}
\end{equation*}
where $\rmH_t \triangleq \rmH(\rvpi(\rvo_t))$. Next, use induction on $t$. For $t=1$, let $c_0 = 0$,
\begin{equation*}
\begin{split}
    \left\| \tau \rvo_{1} - \hat{\rvr}_0 + c_0 \rvone \right\| = \left\| \rvzero \right\| = 0 \le C \sqrt{K}.
\end{split}
\end{equation*}
Assume $\left\| \tau \rvo_{t} - \hat{\rvr}_{t-1} + c_{t-1} \rvone \right\| \le \frac{C \sqrt{K}}{t}$ for some $t \ge 1$. There  are two different cases of the update.

\textbf{First case:} $\rvr_t - \tau \rvo_t = a_t \rvone$, for some $a_t \in \sR$. This means $\rvo_{t+1} = \rvo_{t}$ because $\rmH_t a_t \rvone = \rvzero$. Let $c_t \triangleq \frac{(t-1)c_{t-1}}{t} +  \frac{a_t}{t}$, we have
\begin{equation*}
\begin{split}
    \left\| \tau \rvo_{t+1} - \hat{\rvr}_t + c_t \rvone \right\| &= \left\| \tau \rvo_{t} - \hat{\rvr}_t + \frac{(t-1)c_{t-1}}{t} \rvone + \frac{a_t}{t} \rvone \right\| \\
    &= \left\| \tau \rvo_{t} - \frac{\rvr_{1:t-1} + \rvr_t}{t-1} \cdot \frac{t-1}{t} + \frac{(t-1)c_{t-1}}{t} \rvone + \frac{a_t}{t} \rvone \right\| \\
    &= \left\| \tau \rvo_{t} - \hat{\rvr}_{t-1} + \frac{\hat{\rvr}_{t-1}}{t} - \frac{\tau \rvo_t + a_t \rvone}{t} + \frac{(t-1)c_{t-1}}{t} \rvone + \frac{a_t}{t} \rvone \right\| \\
    &= \left\| \frac{t-1}{t} \cdot \left( \tau \rvo_{t} - \hat{\rvr}_{t-1} + c_{t-1} \rvone \right) \right\| \\
    &\le \frac{t-1}{t} \cdot \frac{C \sqrt{K}}{t} \\
    &\le \frac{C \sqrt{K}}{t+1}.
\end{split}
\end{equation*}

\textbf{Second case:} $\rvr_t - \tau \rvo_t \not= a_t \rvone$, for any $a_t \in \sR$. Then $\rvo_{t+1} \not= \rvo_t$. By the inductive hypothesis,
\begin{equation*}
\begin{split}
    \left| \tau \rvo_{t}(i) - \hat{\rvr}_{t-1}(i) + c_{t-1} \right| \le \left\| \tau \rvo_{t} - \hat{\rvr}_{t-1} + c_{t-1} \rvone \right\| \le \frac{C \sqrt{K}}{t}, \quad \forall i \in [K].
\end{split}
\end{equation*}
Denote $i_{\min} \triangleq \argmin\limits_{i \in [K]}{\rvo_t(i)} $, $i_{\max} \triangleq \argmax\limits_{i \in [K]}{\rvo_t(i)} $, and $\gF(\rvo_t) \triangleq \log{\sum\limits_{i = 1}^K{\exp\left\{ \rvo_t(i) \right\}} } \le \rvo_t(i_{\max}) + \log{K}$. The smallest probability of $\rvpi_t$ can be lower bounded as
\begin{equation*}
\begin{split}
    \min\limits_{i \in [K]}{\rvpi_t(i)} &= \min\limits_{i \in [K]}{ \exp\left\{ \rvo_t(i) - \gF(\rvo_t) \right\} } \\
    &= \exp\left\{ \min\limits_{i \in [K]}{\rvo_t(i)} - \gF(\rvo_t) \right\} \\
    &\ge \exp\left\{ \rvo_t(i_{\min}) - \rvo_t(i_{\max}) - \log{K} \right\} \\
    &\ge \exp\left\{ \frac{\hat{\rvr}_{t-1}(i_{\min})}{\tau} - \frac{C \sqrt{K}}{\tau t} - \frac{c_{t-1}}{\tau} - \frac{\hat{\rvr}_{t-1}(i_{\max})}{\tau} - \frac{C \sqrt{K}}{\tau t} + \frac{c_{t-1}}{\tau} - \log{K} \right\} \\
    &\ge \exp\left\{ - \frac{1}{\tau} - \log{K} - \frac{2 C \sqrt{K}}{\tau t} \right\} \\
    &= \frac{1}{ K \cdot \exp\left\{ \frac{1}{\tau} \right\} } \cdot \exp\left\{ - \frac{2 C \sqrt{K}}{\tau t} \right\} \\
    &\ge \frac{1}{K \cdot \exp\left\{ \frac{1}{\tau} \right\}} \cdot \left( 1 - \frac{2 C \sqrt{K}}{\tau t} \right).
\end{split}
\end{equation*}
Using the choice of learning rate $\eta$ at step $t$,
\begin{equation*}
\begin{split}
    \eta \cdot \lambda_{\min}^{+}\left\{ \rmH_t \right\} &\ge \eta \cdot \min\limits_{i \in [K]}{\rvpi_t(i)} \\
    &\ge \eta \cdot \frac{1}{K \cdot \exp\left\{ \frac{1}{\tau} \right\}} \cdot \left( 1 - \frac{2 C \sqrt{K}}{\tau t} \right) \\
    &\ge \frac{1}{t+1} + \frac{1}{(t+1)(t+C)},
\end{split}
\end{equation*}
where $\lambda_{\min}^{+}\left\{ \rmH_t \right\}$ refers to the smallest positive eigenvalue of $\rmH_t$. Let $c_{t} = c_{t-1}$. By the update,
\begin{equation*}
\begin{split}
    \tau \rvo_{t+1} - \hat{\rvr}_t + c_{t} \rvone &= \tau \rvo_{t} - \hat{\rvr}_{t-1} + c_{t-1} \rvone + \hat{\rvr}_{t-1} - \hat{\rvr}_t - \eta \cdot \rmH_t \left( \tau \rvo_{t} - \hat{\rvr}_{t-1} + c_{t-1} \rvone \right) - \eta \cdot \rmH_t \left( \hat{\rvr}_{t-1} - \rvr_t \right).
\end{split}
\end{equation*}
Denote $\rvdelta_{t} \triangleq \tau \rvo_{t} - \hat{\rvr}_{t-1} + c_{t-1} \rvone$. Note that
\begin{equation*}
\begin{split}
    \hat{\rvr}_{t-1} - \hat{\rvr}_t &= \frac{\rvr_{1:t-1}}{t-1} - \frac{\rvr_{1:t-1} + \rvr_t}{t} = \frac{\rvr_{1:t-1} - (t-1) \cdot \rvr_t }{t(t-1)} = \frac{1}{t} \cdot \left( \hat{\rvr}_{t-1} - \rvr_t \right).
\end{split}
\end{equation*}
We have the following iterative relation,
\begin{equation*}
\begin{split}
    \rvdelta_{t+1} = \left[ \rmI - \eta \cdot \rmH_t \right] \rvdelta_t + \left[ \frac{1}{t} \cdot \rmI - \eta \cdot \rmH_t \right] \left( \hat{\rvr}_{t-1} - \rvr_t \right).
\end{split}
\end{equation*}
By the triangle and H{\" o}lder's inequality, assuming $\rvdelta_t \not= b_t \rvone$ and $\hat{\rvr}_{t-1} - \rvr_t \not= d_t \rvone$ for any $b_t, d_t \in \sR$,
\begin{equation*}
\begin{split}
    \left\| \rvdelta_{t+1} \right\| &\le \lambda_{\max}\left\{ \rmI - \eta \cdot \rmH_t \right\} \cdot \left\| \rvdelta_{t} \right\| + \lambda_{\max}\left\{ \frac{1}{t} \cdot \rmI - \eta \cdot \rmH_t \right\} \cdot \left\| \hat{\rvr}_{t-1} - \rvr_t \right\| \\
    &\le \left( 1 - \eta \cdot \lambda_{\min}^{+}\left\{ \rmH_t \right\} \right) \cdot \left\| \rvdelta_{t} \right\| + \left( \frac{1}{t} - \eta \cdot \lambda_{\min}^{+}\left\{ \rmH_t \right\} \right) \cdot \sqrt{K} \\
    &\le \left( 1 - \eta \cdot \lambda_{\min}^{+}\left\{ \rmH_t \right\} \right) \cdot \frac{C \sqrt{K}}{t} + \left( \frac{1}{t} - \eta \cdot \lambda_{\min}^{+}\left\{ \rmH_t \right\} \right) \cdot \sqrt{K} \\
    &= \frac{C \sqrt{K}}{t} + \frac{\sqrt{K}}{t} - \eta \cdot \lambda_{\min}^{+}\left\{ \rmH_t \right\} \cdot \left( \frac{C \sqrt{K}}{t} + \sqrt{K} \right) \\
    &\le \frac{C \sqrt{K}}{t} + \frac{\sqrt{K}}{t} - \left[ \frac{1}{t+1} + \frac{1}{(t+1)(t+C)} \right] \cdot \left( \frac{C \sqrt{K}}{t} + \sqrt{K} \right) \\
    &= \frac{C \sqrt{K}}{t+1},
\end{split}
\end{equation*}
If either $\rvdelta_t = b_t \rvone$ or $\hat{\rvr}_{t-1} - \rvr_t = d_t \rvone$, for some $b_t, d_t \in \sR$, the result will be similar. If both $\rvdelta_t = b_t \rvone$ and $\hat{\rvr}_{t-1} - \rvr_t = d_t \rvone$, then
\begin{equation*}
\begin{split}
    \tau \rvo_t - \rvr_t &= \tau \rvo_{t} - \hat{\rvr}_{t-1} + c_{t-1} \rvone + \hat{\rvr}_{t-1} - \rvr_t - c_{t-1} \rvone \\
    &= (b_t + d_t - c_{t-1}) \rvone,
\end{split}
\end{equation*}
which reduces to the first case.
\end{proof}

\begin{lem}
Given any $\rvpi \in \Delta$, define $\rmH\left( \rvpi \right) \triangleq \Delta \left( \rvpi \right) - \rvpi \rvpi^\top$. Sorting $\rvpi$ in ascending order,
\begin{equation*}
\begin{split}
    &\quad \rvpi(1) = \rvpi(2) = \cdots = \rvpi(|\gS_1|) \\
    &< \rvpi(|\gS_1|+1) = \rvpi(|\gS_1|+2) = \cdots = \rvpi(|\gS_1| + |\gS_2|) \\
    &\ \ \vdots \\
    &< \rvpi(K - |\gS_k| + 1) = \rvpi(K - |\gS_k| + 2) = \cdots = \rvpi(K),
\end{split}
\end{equation*}
where $\gS_i \triangleq \left\{ \rvpi(|\gS_{<i}| + 1), \rvpi(|\gS_{<i}| + 2), \dots,  \rvpi(|\gS_{<i}| + |\gS_i|) \right\}$, $\forall 1 \le i \le k$, and $|\gS_{<i}| \triangleq \sum\limits_{j < i}{|\gS_j|}$ satisfy \begin{equation*}
\begin{split}
    \bigcup_{i=1}^{k}{\gS_i} &= \left\{ \rvpi(1), \rvpi(2), \dots, \rvpi(K) \right\} \\
    \gS_i \cap \gS_j &= \emptyset, \quad \forall i \not= j.
\end{split}    
\end{equation*}
Then the eigenvalues of $\rmH\left( \rvpi \right)$ are
\begin{equation*}
\begin{split}
    \lambda_{1} &= 0 \\
    \lambda_{2} = \lambda_{3} = \cdots = \lambda_{|\gS_1|} &= \rvpi(|\gS_1|) \\
    \lambda_{|\gS_1| + 1 } &\in \left( \rvpi(|\gS_1|), \rvpi(|\gS_2|) \right) \\
    \lambda_{|\gS_1| + 2} = \lambda_{|\gS_1| + 3} = \cdots = \lambda_{|\gS_1| + |\gS_2|} &= \rvpi(|\gS_2|) \\
    \lambda_{|\gS_1| + |\gS_2| + 1 } &\in \left( \rvpi(|\gS_2|), \rvpi(|\gS_3|) \right) \\
    &\ \ \vdots \\
    \lambda_{K - |\gS_k| +1} &\in \left( \rvpi(K-1), \rvpi(K) \right) \\
    \lambda_{K - |\gS_k| + 2} = \lambda_{K - |\gS_k| + 3} = \cdots = \lambda_{K} &= \rvpi(K).
\end{split}
\end{equation*}
And for any eigenvalue $\lambda$ of $\rmH(\rvpi)$ such that $\lambda \not= \rvpi(i)$, $\forall i \in [K]$, we have
\begin{equation*}
    \sum\limits_{i=1}^{K}{\frac{\rvpi(i)^2}{\rvpi(i) - \lambda}} = 1.
\end{equation*}
\end{lem}
\begin{proof}
Suppose $\lambda$ is an eigenvalue of $\rmH(\rvpi)$ and $\rvv$ is the corresponding eigenvector, i.e.,
\begin{equation*}
    \rmH(\rvpi) \rvv = \left[ \Delta(\rvpi) - \rvpi\rvpi^\top \right] \rvv = \lambda \rvv.
\end{equation*}
It is easy to check that $\rvv = \rvone$ is the eigenvector of $\lambda_0 = 0$. Next, assuming $\lambda \not= 0$, we have
\begin{equation*}
    \rvpi(i) \left( \rvv(i) - \rvpi^\top \rvv \right) = \lambda \rvv(i), \quad \forall i \in [K].
\end{equation*}
Note if $\rvpi^\top \rvv = 0$, we have $\rvpi(i) \rvv(i) = \lambda \rvv(i)$, $\forall i \in [K]$. Since $\rvv \not= \rvzero$, assume $\rvv(j) \not= 0$, then we have $\rvpi(j) = \lambda$. Suppose $\rvpi(j) \in \gS$
\end{proof}

\begin{lem}
Let $\rvpi_1 \triangleq \softmax(\rvo_1)$ and $\rvpi_2 \triangleq \softmax(\rvo_2)$. Then
\begin{equation*}
    \left\| \rvpi_1 - \rvpi_2 \right\|_\infty \le  2 \left\| \rvo_1 - \rvo_2 \right\|_\infty.
\end{equation*}
\end{lem}